\documentclass[11pt,a4paper,twoside]{article}

\usepackage[utf8]{inputenc}
\usepackage[T1]{fontenc}
\usepackage{xfrac}
\usepackage{amsmath}
\usepackage{amssymb}
\usepackage[italian]{babel}

\pagestyle{headings}

% solo nel caso mi decida a mettere il testo in doppia colonna:
%\usepackage[top=60pt,bottom=40pt,left=20pt,right=20pt]{geometry}

\setlength{\columnsep}{20pt}

\newtheorem{definition}{Definizione}
\newtheorem{theorem}{Teorema}
\newtheorem{corollary}{Corollario}[theorem]
\newtheorem{ex}{Esempio}

% in questo modo \emptyset diventa un (bel) simbolo per l'insieme
% vuoto.
\let\emptyset\varnothing%

% stessa storia per epsilon e varepsilon
\let\epsilon\varepsilon%

% idem per i confronti
\let\leq\leqslant%
\let\geq\geqslant%

% per avere il font \mathbbm, che ci permette di avere un \R più bello
% del `semplice' \mathbb
\usepackage{bbm}

\newcommand\N{\mathbb{N}}
\newcommand\R{\mathbbm{R}}
\newcommand\B{\mathcal{B}}
\newcommand\F{\mathcal{F}}
\newcommand\D{\mathcal{D}}

% per gli integrali, aggiunge un po' di spazio prima del `dx' o del
% `dy'.
\newcommand\dx{\,dx}
\newcommand\dy{\,dy}
\newcommand\du{\,du}
\newcommand\dt{\,dt}
\newcommand\dz{\,dz}

% la barra verticale per il ``calcolato su''
\newcommand{\computedat}[1]{\:\bigg\rvert_{#1}}

% una sommatoria ``piccina''
\newcommand\smallsum{\textstyle\sum}

% per rendere più leggibili alcuni passaggi
\newcommand\fulfillment{\longrightarrow}

% Cebyshev è un nome che non posso scrivere con la mia tastiera
\newcommand\Cebyshev{\v Cebyshev}

% cercando di rendere alcune sezioni più semantiche
\newcommand{\explain}[2]{\underbrace{#1}_{\parbox{\widthof{\ensuremath{#1}}}{\footnotesize\raggedright #2}}}

\DeclareMathOperator{\Var}{Var}
\DeclareMathOperator{\Cov}{Cov}
\DeclareMathOperator{\E}{E}
%\DeclareMathOperator{\exp}{exp}

% Funzione di Sopravvivenza
\newcommand\FS{\bar{F}}

\usepackage{mathtools}

% definisci \floor, \floor*, \ceil e \ceil
\DeclarePairedDelimiter\ceil{\lceil}{\rceil}%
\DeclarePairedDelimiter\floor{\lfloor}{\rfloor}

% augmenta \abs e \norm
\DeclarePairedDelimiter\abs{\lvert}{\rvert}%
\DeclarePairedDelimiter\norm{\lVert}{\rVert}%
% inverti la definizione di \abs* e \norm* in modo che \abs e \norm
% cambino la dimensione delle parentesi e la versione con * no.
\makeatletter
\let\oldabs\abs
\def\abs{\@ifstar{\oldabs}{\oldabs*}}
%
\let\oldnorm\norm
\def\norm{\@ifstar{\oldnorm}{\oldnorm*}}
\makeatother

\author{Omar Polo}
\date{\today}
\title{Appunti di CPS}

\usepackage{hyperref}
\hypersetup{
  pdfauthor={Omar Polo},
  pdftitle={Appunti di CPS},
  pdfkeywords={appunti,statistica,probabilità},
  pdfsubject={},
  pdfcreator={pdflatex},
  pdflang={Italian}
}

% nessun rientro per i paragrafi, ma un maggior spazio verticale
\setlength{\parindent}{0em}
\setlength{\parskip}{0.5em}

\begin{document}

\maketitle
\tableofcontents
\thispagestyle{empty}
\newpage
\setcounter{page}{1}

\section{Parti mancanti}
\begin{itemize}
\item capitoli \([1, 5]\)
\item capitolo 13: ``Leggi di v.c.\ trasformate''
\item capitolo 15:
  \begin{itemize}
  \item indici di posizione e variabilità per v.c.\ multivariate
  \item varianza di una combinazione lineare
  \item varianza di correlazione lineare
  \end{itemize}
\end{itemize}

\section{Variabili casuali}

\subsection{Nel Discreto}

\subsubsection{Bivariate con legge discreta}

Una v.c.\ bivariata con legge discreta è definita da:
\begin{itemize}
\item il supporto congiunto \(S_{X,Y} = \{ (x_i,y_i) \in \R^2, i\in I
  \subseteq \N \}\) successione finita o numerabile di punti distinti
  in \(\R^2\) senza punti di accumulazione all'infinito.
\item la f.m.p\ congiunta \(p_{X,Y} : S_{X,Y}\rightarrow [0,1]\)
\end{itemize}

La legge congiunta della v.c.\ \((X,Y)\) è allora, per ogni
\(B\in\B_2\) data da:
\begin{equation}\label{eq:legge-congiunta-bivariate}
  P_{X,Y}(B) = P((X,Y) \in B) = \sum_{(x,y)\in B\cap S_{X,Y}} p_{X,Y}(x,y)
\end{equation}

\paragraph{Leggi marginali}
Da una v.c.\ bivariata con legge discreta \((X,Y)\) specificata da
\(S_{X,Y}\) e \(p_{X,Y}(x, y)\) si possono ``estrarre'' varie leggi di
v.c.\ univariata. Anzitutto conviene considerare le leggi delle
componenti \(X\) e \(Y\), dette \textbf{leggi marginali}.

La legge marginale di \(X\) ha:
\begin{itemize}
\item supporto marginale
  \[ S_X = {x\in\R : (x,y) \in S_{X,Y}\mbox{ per qualche } y\in\R} \]
\item f.m.p.\ marginale
  \[
    p_X(x) = P(X=x) = \sum_{y:(x,y}\in S_{X,Y} p_{X,Y}(x,y), \mbox{
      per } x\in S_X .
  \]
\end{itemize}

Similmente si può ottenere la legge marginale di \(Y\) in modo del
tutto simmetrico.

\paragraph{Leggi condizionali}
Data una v.c.\ bivariata con legge discreta \((X,Y)\) si possono
``estrarre'' anche due famiglie di leggi condizionali: la
\textbf{legge condizionale di \(Y\) dato un valore osservabile di
  \(X\)} indicata con \(Y|X=x, x\in S_X\):
\begin{itemize}
\item supporto condizionale
  \[ S_{Y|X=x} = { y\in\R : (x,y) \in S_{X,Y} } \]
\item f.m.p.\ condizionale
  \begin{align*}
      P_{Y|X=x}(y) &= P(Y=y| X=x) = \frac{P(X=x, Y=y)}{P(X=x)} \\
                   &= \frac{p_{X,Y}(x,y)}{p_X(x)} \quad\mbox{per } y\in S_{Y|X=x} .
  \end{align*}
\end{itemize}
e la legge condizionale di \(X\) dato un valore osservabile di \(Y\)
che si deduce in modo del tutto analogo.

\paragraph{Componenti indipendenti e dipendenti}
\begin{definition}
  La v.c.\ con legge discreta \((X,Y)\) si dice con componenti
  indipendenti se
  \[ S_{X,Y} = S_X \times S_Y \]
  e se per ogni \((x,y)\in S_{X,Y}\) si ha
  \[ p_{X,Y}(x,y) = p_X(x) p_Y(y) . \]
\end{definition}

Se invece le componenti di \(X\) e \(Y\) di \((X,Y)\) non sono
indipendenti si dice che la v.c.\ ha componenti dipendenti.
\subsection{Nel Continuo}

\subsubsection{Supporto di una v.c.\ con legge continua}
Il supporto \(S_X\) di una v.c.\ univariata \(X\) con legge continua è
il più piccolo sottoinsieme chiuso di \(\R\) al quale \(P_X\) da
probabilità 1. Si ricorda che un insieme è chiuso se contiene tutti i
suoi punti di accumulazione.  Quindi \(S_X\in\B_1\) è tale che
\begin{enumerate}
\item \(S_X\) è chiuso
\item \(S_X \subseteq C\) per ogni \(C\subseteq \R\) chiuso con
  \(P_X(C) = P(X\in C) = 1\)
\end{enumerate}

In altre parole, \(S_X\) è la chiusura dell'insieme \({ x\in\R :
  p_X(x) > 0 }\).  Si ricorda che la chiusura di un insieme numerico è
l'unione fra l'insieme stesso e i punti non dell'insieme che sono però
suoi punti di accumulazione.

\subsubsection{Bivariate con legge continua}
\((X,Y)\) è una v.c.\ bivariata con legge continua se per ogni
\(B\in\B_2\), con \(B = [a_1,b_1]\times [a_2,b_2]\), si ha:
\begin{align*}
  P_{(X,Y)}(B) &= P((X,Y)\in B) \\
               &= P(a_1 \leq X \leq b_1, a_2\leq Y\leq b_2) \\
               % `iint' è il doppio-integrale, \int\int non va
               % bene. certe volte LaTeX è proprio strano, cosa mi hai
               % combinato, Lamport??
               &= \iint_B p_{(X,Y)}(x,y)\dx\dy
\end{align*}
dove \(p_{X,Y}(x,y)\) ha le proprietà di
\begin{itemize}
\item non negatività, per ogni \((x,y) \in \R^2\)
\item normalizzazione: \(\iint_{\R^2} p_{X,Y}(x,y) \dx\dy = 1\).
\end{itemize}

Per una v.c.\ con legge continua vale che \(P(X=x, Y=y)\) sia zero
per ogni \((x,y)\in\R^2\) e che \(S_{X,Y}\) sia la chiusura
di \(\{ (x,y)\in\R^2 \mid p_{X,Y} (x,y) > 0 \}\).

Come nel caso discreto, si possono ottenere le leggi univariate
indotte, che saranno di tipo continuo:
\begin{itemize}
\item marginale di \(X\) con f.d.p.
  \[ p_X(x) = \int_{-\infty}^{+\infty}p_{X,Y}(x,y)\dy \]
\item marginale di \(Y\) con f.d.p.
  \[ p_Y(y) = \int_{-\infty}^{+\infty}p_{X,Y}(x,y)\dx \]
\item condizionale \(Y|X=x, x\in S_X\) con f.d.p.
  \[ p_{Y|X=x}(y) = \frac{p_{X,Y}(x,y)}{p_X(x)} \]
\item condizionale \(X|Y=y, y\in S_Y\) con f.d.p.
  \[ p_{X|Y=y}(x) = \frac{p_{X,Y}(x,y)}{p_Y(y)} \]
\end{itemize}

Infine, i supporti si ottengono con le identiche formule del caso
discreto:
\begin{align*}
  S_X      &= \{ x\in\R\mid (x,y)\in S_{X,Y} \mbox{ per qualche } y \} \\
  S_Y      &= \{ y\in\R\mid (x,y)\in S_{X,Y} \mbox{ per qualche } x \} \\
  S_{Y|X=x} &= \{ x\in\R\mid (x,y)\in S_{X,Y} \} \\
  S_{X|Y=y} &= \{ y\in\R\mid (x,y)\in S_{X,Y} \} \\
\end{align*}

\paragraph{Componenti indipendenti}
In modo del tutto analogo al caso continuo, si dice che una v.c.\
bivariata \((X,Y)\) con legge continua ha componenti indipendenti se
per ogni \((x,y)\in\R^2\) vale che
\[ p_{X,Y}(x,y) = p_X(x) p_Y(y). \]

\section{Indici di posizione}

\subsection{Moda}
\begin{definition}
  Sia \(X\) una v.c.\ con legge discreta o continua e f.m.p./f.d.p.\
  \(p_X(x)\).  Si dice \textbf{moda} di \(X\), indicata con
  \(x_{mo}\), un valore del supporto di \(X\) per cui
  \[
    p_X(x_{mo}) \geq p_X(x) \quad \forall x\in S_X
  \]
  
  Nel caso del continuo si richiede inoltre che la densità di \(X\)
  sia continua almeno da destra o da sinistra in \(x_{mo}\).
\end{definition}

\subsection{Mediana}
\begin{definition}
  Sia \(X\) una v.c.\ univariata con f.r.\ \(F_X(x)\). Si dice
  \textbf{mediana} di \(X\), indicata con \(x_{0.5}\), un valore reale
  tale che valgano simultaneamente
  \[
    P(X \leq x_{0.5}) \geq 0.5
    \quad\mbox{e}\quad
    P(X \geq x_{0.5}) \geq 0.5
  \]
\end{definition}

La mediana non è necessariamente unica. Tutte le soluzioni
dell'equazione \(F_X(x) = 0.5\) sono mediane di \(X\). Se invece
l'equazione non ha soluzioni, la mediana di \(X\) è unica e risulta
essere il più piccolo valore di \(x\) per cui \(F_X(x) \geq 0.5\).

\subsection{Quantile-\(p\)}
\begin{definition}
  Sia \(X\) una v.c.\ univariata con f.r.\ \(F_X(x)\).  Per \(p\in
  (0,1)\) si dice \textbf{quantile}-\(p\) di \(X\), indicato con
  \(x_p\), un valore reale tale che valgano simultaneamente:
  \[
    P(X \leq x_P) \geq p
    \quad\mbox{e}\quad
    P(X \geq X_p) \geq 1 - p.
  \]
\end{definition}

Si tratta di una generalizzazione del concetto di mediana, infatti la
mediana è anche detta quantile-0.5.

Come la mediana, anche il quantile-\(p\) non è necessariamente unico.

\subsection{Valore atteso}
Il valore atteso \(\E(X)\) è la media aritmetica ponderata dei valori
assumibili dalla v.c.\ con pesi dati dalla f.m.p. Se la variabile ha
supporto continuo, la ponderazione è data dalla f.d.p.\ e la somma
viene intesa in senso continuo, ovvero come un integrale.

\(\E(X)\) è quindi definita come
\[
  \E(X) = \begin{dcases}
    \sum_{x\in S_X} x p_X(x) &\mbox{se \(X\) ha legge discreta} \\
    \int_{-\infty}^{+\infty} x p_X(x) \dx &\mbox{se \(X\) ha legge continua}
  \end{dcases}
\]

Si richiede che la somma (o l'integrale) convergano assolutamente,
quindi:
\[
  \sum_{x\in S_X} \abs{x} p_X(x) < +\infty \quad\mbox{oppure}\quad
  \int_{-\infty}^{+\infty}\abs{x} p_X(x) dx < +\infty
\]

\subsubsection{Proprietà del valore atteso}

\paragraph{Valore atteso di trasformate} Siano \(X\) e \(Y\) v.c.\
univariate con \(Y = g(X)\), allora vale che
\[
  \E(Y) = \E(g(X)) = \begin{dcases}
    \sum_{x\in S_X} g(x) p_X(x) &\mbox{se \(X\) ha legge discreta} \\
    \int_{-\infty}^{+\infty} g(x) p_X(x) \dx &\mbox{se \(X\) ha legge continua}
  \end{dcases}
\]

\paragraph{Valore atteso del prodotto di v.c.\ indipendenti}
Sia \(X = (X_1, \dots, X_d)\) una v.c.\ con componenti indipendenti,
allora
\[
  \E(X_1, \dots, X_d) = \prod_{i=1}^d \E(X_i)
\]

\paragraph{Proprietà di Cauchy} Quando esiste finito, il valore atteso
può non essere un punto del supporto di \(X\), ma e sempre intermedio
fra i punti del supporto.  Supponendo, senza perdita di generalità che
\(S_X = \{ x_1, \dots, x_k \}\)
\[
  x_1 < x_2 < \cdots < x_k
\]
si ha
\[
  x_1 \leq x_i \leq x_k \quad \forall i \in \{1, 2, \dots, k\}
\]
e quindi
\[
  x_1 p_X(x_i) \leq x_i p_X(x_i) \leq x_k p_X(x_i)
  \quad\forall i \in \{1, 2, \dots, k\}.
\]

Sommando le diseguaglianze si ottiene
\[
  x_1 \sum_{i=1}^k p_X(x_i) \leq \sum_{i=1}^k x_i p_X(x_i) \leq x_k
  \sum_{i=1}^k p_X(x_i)
\]
da cui, per la normalizzazione, si ottiene
\[
  x_1 \leq \E(X) \leq x_k .
\]

\paragraph{Proprietà di linearità} Siano \(X\) e \(Y\) v.c.\
univariate con \(Y = aX + b\), allora
\[
  \E(Y) = \E(aX + b) = a\E(X) + b.
\]

\paragraph{Proprietà di linearità generalizzata} Se \((X, Y)\) è una
v.c.\ bivariata e \(T = aX + bY\) una combinazione lineare delle
componenti di \((X, Y)\) con \(a, b \in \R\), allora
\[
  \E(t) = \E(aX + bY) = a\E(X) + b\E(Y).
\]

\paragraph{Proprietà del baricentro} Si tratta di un caso particolare
della linearità
\[
  \E(X - \E(X)) = 0.
\]

\paragraph{Proprietà dei minimi quadrati} Se \(X\) è una v.c.\
univariata e i valori attesi indicati esistono, allora per ogni
\(c\in\R\)
\[
  \E\left( (X-c)^2 \right) \geq \E\left( (X - \E(X))^2 \right)
\]
dove
\begin{itemize}
\item \(c\) è una predizione puntuale della realizzazione futura di
  \(X\)
\item \(X-c\) è l'errore di predizione
\item \( (X-c)^2 \) è la perdita quadratica dovuta all'errore di
  predizione
\item \(\E((X-c)^2)\) è la perdita quadratica media, detta rischio quadratico
\end{itemize}

\section{Indici di variabilità}
Sia \(X\) una v.c.\ univariata con legge discreta e supporto finito.
Si possono cogliere aspetti importanti della distribuzione di \(X\)
attraverso indici sintetici.

\subsection{Varianza}
L'indice di variabilità principale, la varianza di \(X\), è definito
come la media aritmetica ponderata del quadrato degli scarti di \(X\)
dal proprio valore atteso.
\[
  \Var(X) = \E\left( (X - \E(X))^2 \right) = \begin{dcases}
    \sum_{x\in S_X}{(x - \E(X))}^2 p_X(x) & \mbox{legge continua} \\
    \int_{-\infty}^{+\infty} {(x - E(X))}^2 p_X(x) \dx & \mbox{legge
      discreta}
  \end{dcases}
\]

\subsubsection{Proprietà}

\paragraph{Non negatività} Per ogni \(X\) con varianza finita
\(Var(X)\geq 0\).  \(\Var(X) = 0\) solo per \(X\sim \D(x_0)\).

\paragraph{Formula per il calcolo}
\[
  \Var(X) = \E(X^2) - ( \E(X) )^2
\]

\paragraph{Invarianza rispetto a traslazioni}
\[
  \Var(X+b) = \Var(X)
\]

\paragraph{Omogeneità di secondo grado}
\[
  \Var(aX) = a^2 \Var(X)
\]

\subsection{Scarto quadratico medio}

Lo scarto quadratico medio, o deviazione standard, è definito come la
radice quadrata aritmetica della varianza.  In simboli
\[
  \sigma_X = \sqrt{\Var(x)}
\]

\subsection{Range}

Il \textit{range} di una v.c.\ univariata \(X\), indicato con \(R_X\)
è definito per variabili con supporto limitato come
\[
  R_X = \max(S_X) - \min(S_X).
\]

\subsection{Scarto interquantilico}
Lo scarto interquantilico di una v.c.\ univariata \(X\), indicato con
\(IQR_X\) è definito come
\[
  IQR_X = x_{0.75} - x_{0.25}
\]

\subsection{Diseguaglianze di Markov e \Cebyshev}
\begin{theorem}[Diseguaglianza di Markov]
  Sia \(X\) una v.c.\ non negativa con valore atteso \(\mu = \E(X) >
  0\) finito.  Allora per ogni \(c > 0\) vale
  \[
    P(X \geq c \mu) \leq \frac 1{c}.
  \]
\end{theorem}

\begin{theorem}[Diseguaglianza di \Cebyshev]
  Sia \(X\) una v.c.\ univariata con valore atteso \(\mu = \E(X)\)
  finito e varianza \(\sigma^2 = \Var(X) > 0\) anch'essa
  finita. Allora, per ogni \(k > 0\) vale
  \[
    P(\abs{X - \mu} \geq k \sigma) \leq \frac 1{k^2} .
  \]
\end{theorem}

Entrambe le diseguaglianze sono poco informative per \(c \in
\left(0,1\right]\) o \(k \in \left(0,1\right] \), ma diventano molto
informative negli altri casi.

\subsection{Covarianza}
La covarianza è un indice sintetico della dipendenza delle componenti
di una v.c.\ bivariata. Se indichiamo il supporto come successione
dipendente di due indici:
\[
  S_{X,Y} = { (x_i,y_i) \quad i\in I, j \in J }
  \mbox{ con \(I\) e \(J\) finiti o numerabili}
\]
ed esprimere la f.m.p.\ in forma abbreviata come applicazione:
\[
  (x_i,y_i) \rightarrow p_{ij} = P(X=x_i, Y=y_i).
\]
Allora la covarianza, indicata con il simbolo \(\Cov(X,Y)\), è
definita come media ponderata del prodotto di scarti:
\begin{align*}
  \Cov(X,Y) &= \sum_{i\in I} \sum_{i\in J} (x_i -
              \E(X))(y_i-\E(Y))p_{ij} \\
            &= \E((X-\E(X)))(Y-\E(Y))
\end{align*}

Per il calcolo della covarianza è nota una formula per il calcolo
analoga a quella che si usa per la varianza
\[
  \Cov(X,Y) = \E(XY) - \E(X)\E(Y) .
\]

\section{Funzione di ripartizione e di sopravvivenza}
\begin{definition}
  Si dice \textbf{funzione di ripartizione} di \(X = (X_1,\dots,X_d)\)
  la funzione
  \[ F_X : \R^d\to [0,1] \]
  che a ciascun punto \(x = (x_1,\dots,x_d)\) di \(\R^d\) fa
  corrispondere il valore d'immagine
  \begin{align*}
    F_X(x) &= P(X_1\leq x_1, \dots, X_d \leq x_d) \\
           &= P_X((-\infty, x_1], \times\cdots\times (-\infty, x_d]) \\
           &= P(\cap_{i=1}^d \{ s\in S \mid X_i(s) \leq x_i \}).
  \end{align*}
\end{definition}

Se \(X\) ha componenti indipendenti, per ogni \(x\in\R^d\) vale la
relazione
\[
  F_X(x) = F_{X_1,\dots,X_d} (x_1, \dots, x_d) = \prod_{i=1}^d F_{X_i}(x_i)
\]

\begin{theorem}[Proprietà strutturali]\label{th:proprietà-strutturali}
  Sia \(X\) una v.c.\ univariata con legge qualsiasi.  La funzione di
  ripartizione \(F_X(x)\) e una applicazione \(F_X : \R\to [0,1]\) che
  gode delle seguenti proprietà:
  \begin{itemize}
  \item è monotona non decrescente
    \[ x_1 < x_2 \Rightarrow F_X(x_1) \leq F_X(x_2) \]
  \item è continua da destra in ogni punto \(x\in\R\)
    \[ \forall x\in\R \quad \lim_{\epsilon\to 0^+} F_X(x+\epsilon) = F_X(x) \]
  \item i limiti agli estremi del dominio sono zero ed uno:
    \begin{align*}
      \lim_{x\to -\infty} F_X(x) = 0 \\
      \lim_{x\to +\infty} F_X(x) = 1
    \end{align*}
  \end{itemize}
\end{theorem}

\begin{theorem}[Caratterizzazione]\label{teorema-di-caratterizzazione}
  Se \(F : \R\to[0,1]\) ha le proprietà 1, 2 e 3 del
  Teorema~\ref{th:proprietà-strutturali} allora esiste una v.c.\
  univariata \(X\) con legge di probabilità \(P_X\) di cui \(F\) ne è
  la funzione di ripartizione.
\end{theorem}

\subsection{Caso univariato}
Se \(X\) è una v.c.\ univariata la sua \(F_X(x)\) permette di
calcolare agevolmente le probabilità degli intervalli.
\[
  P(a < X \leq b) = F_X(b) - F_X(a) \quad\mbox{con } a < b
\]

\paragraph{Funzione di sopravvivenza} La funzione \(P(X>x)\) viene
detta funzione di sopravvivenza di \(X\)
\[
  P(X>x) = 1 - P(X\leq X) = 1 - F_X(x).
\]

\paragraph{Dalla \(p_X(x)\) alla \(F_X(x)\)} Sia \(X\) una v.c.\
univariata.  Dalla definizione si ha subito che
\[
  F_X(x) = P(X\leq x) = \begin{dcases}
    \sum_{t\in S_X \mid t\leq x} p_X(t) & \mbox{se \(X\) ha legge discreta} \\
    \int_{-\infty}^x p_X(t) dt & \mbox{se \(X\) ha legge continua}
  \end{dcases}
\]

Si possono quindi fare le seguenti osservazioni:
\begin{itemize}
\item se \(X\) ha legge discreta univariata, la \(F_X(x)\) è costante
  a tratti, con punti di salto posizionati ai punti del supporto, e
  con valore del salto pari alla massa di probabilità posta sul punto;
\item se \(X\) ha legge continua, la sua funzione di ripartizione è
  una funzione continua di \(x\in\R\), in quanto primitiva di una
  funzione integrabile.
\end{itemize}

Le varie \(F_X(x)\) calcolate verranno mostrate nelle sezioni delle
leggi.

\paragraph{Dalla \(F_X(x)\) alla funzione di massa di probabilità}
Data la \(F_X(x)\) di una v.c.\ discreta \(X\) si può recuperare il
supporto come insieme dei punti di salto
\[
  S_X = \{ x\in\R \mid F_X(x) - \lim_{\epsilon\to 0+}F_X(x-\epsilon)>0 \}
\]
da cui poi dedurre
\[
  p_X(x) = P(X = x) = \lim_{\epsilon\to 0^+} P(x-\epsilon < X \leq x)
  = F_X(x) - \lim_{\epsilon\to 0^+} F_X(x-\epsilon)
\]

\paragraph{Dalla \(F_X(x)\) alla funzione di densità di probabilità}
Nei punti \(x\) in cui \(p_X(x)\) è continua, la \(F_X(x)\) è
derivabile e vale
\[
  p_X(x) = \frac{d}{dx}F_X(x).
\]

\section{Funzione generatrice dei momenti}
\begin{definition}[Momenti]
  Si dicono momenti di una v.c.\ univariata \(X\) i valori
  \[
    \mu_r = \E(X^r), r = 1, 2, \dots
  \]

  In particolare, \(\mu_r\) è detto \textbf{momento \(r\)-esimo} di \(X\).
\end{definition}

Si ha subito che:
\begin{align*}
  \mu_1 &= \mu = \E(X) \\
  \mu_2 - \mu_1^2 &= \E(X^2)- \left( \E(X) \right)^2 = \Var(X)
\end{align*}

\begin{definition}[Funzione generatrice dei momenti]
  Sia \(X\) una v.c.\ univariata con f.m.p.\ o f.d.p.\ \(p_X(x)\).  La
  funzione generatrice dei momenti di \(X\), indicata con \(M_X(t)\),
  è una funzione di variabile reale definita da
  \[
    M_X(t) = \E(e^{tX}) = \begin{dcases}
      \sum_{x\in S_X} e^{tx} p_X(x) &\mbox{se \(X\) ha legge discreta} \\
      \int_{-\infty}^{+\infty} e^{tx} p_X(x) \dx &\mbox{se \(X\) ha
        legge continua}
    \end{dcases}
  \]
\end{definition}

Tale funzione gode delle seguenti proprietà:
\begin{itemize}
\item nell'origine vale sempre \(M_X(0) = 1\) per ogni \(X\);
\item il dominio di finitezza di \(M_X(t)\), ovvero \(D_X = \{ t\in\R
  \mid M_X(t) < +\infty \}\), è \textbf{convesso} (ovvero è un
  intervallo, una semiretta o l'intera retta reale);
\item \(M_X(t) > 0 \quad\forall t\in D_X\)
\end{itemize}

\begin{definition}[Funzione generatrice dei momenti propria]
  Si dice che la v.c.\ univariata \(X\) ha funzione generatrice dei
  momenti propria se il dominio di finitezza di \(M_X(t)\) include
  l'origine come punto interno.
\end{definition}

Una proprietà interessa delle funzioni generatrici dei momenti proprie
è che nell'origine hanno derivate di ogni ordine, i cui valori
``generano'' i momenti di \(X\)
\[
  \mu_r = \E(X^r) = \frac{d^r}{dt^r} M_X(t) \computedat{t = 0} = M_X^{(r)}(0).
\]
che rende in alcuni casi più semplice calcolare valore atteso e varianza.

\begin{definition}(Generatrice della somma di v.c. indipendenti) Sia
  \(X = (X_1, \dots, X_d)\) una v.c.\ multivariata con componenti
  \(X_i\) indipendenti che hanno funzione generatrice dei momenti
  propria \(M_{X_i}(t)\).  Allora \(S = \sum_{i=1}^d X_i\) ha funzione
  generatrice dei momenti propria
  \[
    M_S(t) = \prod_{i=1}^d M_{X_i}(t).
  \]
\end{definition}

\section{Leggi di tipo discreto}
Le leggi discrete sono in generale individuate da due ingredienti:
\begin{enumerate}
\item un insieme senza punti di accumulazione al finito
  \[
    S_X = \cup_{x\in I}{x_i}, \quad x_i\in\R^d,\quad i\in I\subseteq \N^+
  \]
  detto \textbf{supporto} della variabile casuale.  
\item una applicazione
  \[
    p_X : S_X \rightarrow [0, 1]
  \]
  detta \textbf{funzione di massa di probabilità} che soddisfi le
  condizioni:
  \begin{itemize}
  \item \(p_X(x) > 0\) per ogni \(x\in S_x\)
  \item \(\sum_{x\in S_x} p_X(x) = 1\)
  \end{itemize}
\end{enumerate}

Dato il supporto e la funzione massa di probabilità, la legge discreta
corrispondente è definita da
\[
  P_X(B) = P(X\in B) = \sum_{x\in S_X \cap B} p_X(x).
\]

\subsection{Leggi degeneri}
Si dice che una v.c. \(d\)-variata \(X\) ha legge degenere in
\(x_0\in \R^d\), valore prefissato, e si scrive \(X\sim \D(x_0)\) se
per \(B\in\B_d\) la legge di probabilità di \(X\) è
\[
  P_X(B) = P(X\in B) = \begin{cases}
    1 & \mbox{se } x_0\in B \\
    0 & \mbox{se } x_0 \not\in B .
  \end{cases}
\]

\paragraph{Funzione di ripartizione}
\[
  F_X(x) = \begin{cases}
    0 & \mbox{se } x < x_0 \\
    1 & \mbox{se } x \geq x_0
  \end{cases}
\]

\paragraph{Generatrice dei momenti, valore atteso e varianza}
\begin{align*}
  M_X(t) &= \E(e^{tX}) = \sum_{x\in S_X} e^{tx} p_X(x) = e^{t x_0} (1)
           = e^{t x_0} \\
  M_X'(t) &= x_0 e^{tx_0} \Rightarrow \E(X) = M_X'(0) = x_0 \\
  M_X''(t) &= x_0^2e^{t x_0} \Rightarrow \E(X^2) = M_X''(0) = x_0^2
\end{align*}
da cui \(\Var(X) = \E(X^2) - (\E(X))^2 = x_0^2 - (x_0)^2 = 0\).

\subsection{Leggi binomiali}
Si dice che la v.c.\ univariata \(X\) ha legge binomiale con indice
\(n \in \N^+\) e parametro \(p\in (0,1)\), e si scrive \(X\sim Bi(n,
p)\), se per ogni \(B \in \B_1\) vale
\[
  P_X(B) = P(X\in B) = \sum_{x\in\{0,1,\dots,n\}\cap B} {n\choose x}p^x(1-p)^{n-x}.
\]

Quando \(n = 1\) le leggi binomiali sono dette di \textbf{di
  Bernoulli} o \textbf{binomiali elementari}.

Alcuni risultati noti:
\[
  F_X(x) = \begin{cases}
    0   & \mbox{se } x < 0 \\
    1-p & \mbox{se } 0\leq x < 1 \\
    1   & \mbox{se } x\geq 1
  \end{cases}\qquad\mbox{supponendo } X\sim Bi(1, p)
\]

\paragraph{Proprietà additiva}
Sia \(X = (X_1, \dots, X_d)\) una v.c.\ multivariata con componenti
\(X_i\) indipendenti e legge marginale \(X_i\sim Bi(n_i, p)\), dove
\(n_i \in \N^+\) e \(p \in (0,1)\) per cui
\[
  M_{X_i}(t) = \left( 1 - p + pe^t\right)^{n_i}
\]
allora \(S = \sum_{i=1}^d X_i\) ha f.g.m.\ propria
\begin{align*}
  M_S(t) &= \prod_{i=1}^d M_{X_i}(t) \\
         &= \prod_{i=1}^d \left( 1-p+pe^t \right)^{n_i} \\
         &= \left( 1 - p + pe^t \right)^{\sum_{i=1}^d n_i}
\end{align*}
per cui
\[
  S \sim Bi\left( \smallsum_{i=1}^d n_i, \: p \right).
\]

\subsection{Leggi uniformi discrete}
La v.c.\ \(d\)-variata \(X\) ha legge uniforme discreta in \(D =
\cup_{i=1}^k{x_i}, x_i\in \R^d\) dove gli \(x_i\) sono \(k\) punti
distinti, e si scrive in breve \(X\sim Ud(X_1,\dots, x_k)\) se
\begin{itemize}
\item \(S_X = D\)
\item \(p_X(
  x) = \sfrac{1}{k}\) per ogni \(x \in S_X\).
\end{itemize}

\subsection{Leggi ipergeometriche}
Si dice che la v.c.\ univariata \(X\) ha legge ipergeometrica con
indice \(n\in\N^+\) e parametri \(N\) e \(D\), dove \(n\leq N \in
N^+\) e \(D\in N^+\) con \(D\leq N\), e si scrive in breve \(X\sim
IG(n; D, N)\) se vale
\[
  P_X(\{x\}) = P(X=x) = \frac{{D\choose x} {N-D\choose n-x}}{{N\choose n}}
\]
per tutti i valori \(x\) per cui hanno senso i coefficienti binomiali
e 0 altrimenti.  Usando il simbolo \(S_x\) per i valori di \(x\) per
cui \(P(X = x) > 0\) le leggi \(P_X\) ipergeometriche sono tali che
per ogni \(B\in\B_1\)
\[
  P_X(B) = P(X\in B) = \sum_{x\in S_x\cap B} \frac{{D\choose x} {N-D\choose n-x}}{{N\choose n}}
\]

\subsection{Leggi di Poisson}

\begin{definition}
  Si dice che \(X\) ha legge di Poisson con parametro \(\lambda > 0\) e
  si scrive \(X\sim P(\lambda)\) se è una v.c.\ univariata con legge
  discreta con supporto \(S_X = \N\) e f.m.p.\ per \(x\in S_X\) pari a
  \[
    p_X(x) = e^{-\lambda} \frac{\lambda^x}{x!} .
  \]
\end{definition}

Per verificare che si tratti di una buona definizione occorre
controllare le solite due condizioni:
\begin{itemize}
\item la positività di \(p_X(x)\) sul supporto \(S_X = \N\) è banale
  perché \(p_X(x)\) è il prodotto di tre fattori positivi.
\item la normalizzazione segue da
  \begin{align*}
    \sum_{x\in S_X} p_X(x)
    &= \sum_{x=0}^{+\infty} e^{-\lambda}
      \frac{\lambda^x}{x!} & \\
    & = e^{-\lambda}\sum_{x=0}^{+\infty} \frac {\lambda^x}{x!} &
                           \mbox{noto: }\sum_{x=0}^{+\infty}\frac{\lambda^x}{x!} = e^{\lambda} \\
    &= e^{-\lambda}e^\lambda = e^{-\lambda + \lambda} = e^0 = 1. & \\
  \end{align*}
\end{itemize}

Si usa tipicamente \(X\sim P(\lambda), \lambda > 0\) per modellare la
distribuzione di una variabile casuale che esprime un
\textbf{conteggio} con supporto illimitato superiormente (o molto più
grande dei valori tipicamente assunti dal conteggio).  È questo il
caso di \(Bi(n,p)\) per \(p = \sfrac{\lambda}{n}\) ed \(n\)
sufficientemente grande.

\paragraph{Valore atteso}
\[
  \E(x) = \sum_{x\in S_X} xp_X(x) = \sum_{x=0}^{+\infty} x
  e^{-\lambda} \frac{\lambda^x}{x!} = \lambda\sum_{x=1}^{+\infty}
  e^{-\lambda} \frac{\lambda^{x-1}}{(x-1)!} = \lambda
  \sum_{x=0}^{+\infty} e^{-\lambda} \frac{\lambda^i}{i!} = \lambda
\]

\paragraph{Funzione generatrice dei momenti e varianza}
\begin{align*}
  M_X(t)
  &= \E(x^{t X}) = \sum_{x\in S_X} e^{tx} p_X(x) \\
  &= \sum_{x=0}^{+\infty} \left( e^t \right)^x e^{-\lambda}\frac{\lambda^x}{x!} \\
  &= e^{-\lambda} \sum_{x=0}^{+\infty} \frac{\left( \lambda e^t\right)^x}{x!} \\
  &= e^{-\lambda} e^{\lambda e^t} \\
  &= e^{\lambda (e^t -1)}
\end{align*}
da cui
\begin{align*}
  M_X'(t) &= e^{\lambda (e^t -1) \lambda e^t} \Longrightarrow \E(X) =
            M_X'(0) = \lambda \\
  M_X''(t) &= e^{\lambda (e^t -1)} \lambda^2 e^{2t} + e^{\lambda (e^t
             -1)}\lambda e^t \Longrightarrow \E(X^2) = M_X''(0) =
             \lambda^2 + \lambda
\end{align*}
e pertanto
\[
  \Var(X) = \E(X^2) - (\E(X))^2 = \lambda^2 + \lambda - (\lambda)^2 = \lambda.
\]

\paragraph{Proprietà additiva}
Sia \(X = (X_1, \dots, X_d)\) una v.c.\ multivariata con componenti
\(X_i\) indipendenti e legge marginale \(X_i \sim P(\lambda_i)\) dove
\(lambda_i > 0\) per \(i = 1, \dots, n\)
\[
  M_{X_i}(t) = e^{\lambda_i (e^t -1)}
\]
allora \(S = \sum_{i=1}^d X_i\) ha f.g.m.\ propria
\begin{align*}
  M_S(t)
  &= \prod_{i=1}^d M_{X_i}(t) \\
  &= \prod_{i=1}^d e^{\lambda_i (e^t -1)} \\
  &= \exp \left\{ \left( \sum_{i=1}^d \lambda_i \right) (e^t-1) \right\}
\end{align*}
da cui si evince che
\[
  S \sim P\left( \smallsum_{i=1}^d \lambda_i \right) .
\]

\subsection{Leggi geometriche}

\begin{definition}
  Si dice che \(X\) ha legge geometrica con parametro \(p\in (0,1)\) e
  si scrive \(X\sim Ge(p)\) se è una v.c.\ univariata con legge
  discreta che ha supporto \(S_X = \N^+\) e f.m.p.\ per \(x\in S_X\)
  pari a
  \[
    p_X(x) = p\times(1-p)^{x-1}.
  \]
\end{definition}

Verifichiamo che sia una buona definizione:
\begin{itemize}
\item positività di \(p_X(x)\) sul supporto \(S_X = \N^+\) banale
  perché è il prodotto di fattori positivi
\item la normalizzazione segue da
  \begin{align*}
    \sum_{x\in S_X} p_X(x)
    &= \sum_{x=1}^{+\infty} p (1-p)^{x-1} \\
    &= p \sum_{x=1}^{+\infty} (1-p)^{x-1} \\
    &= p \sum_{i=0}^{+\infty} (1-p)^i \\
    &= p \frac{1}{1-(1-p)} = 1.
  \end{align*}
  dove si è usata il risultato della serie geometrica con ragione
  \(x\), \(\abs{x} < 1\):
  \begin{align*}
    \sum_{i=0}^\infty x^i &= \lim_{n\to\infty}\sum_{i=0}^n x^i \\
                          &= \lim_{n\to\infty}\frac{1-x^{n+1}}{1-x} \\
                          &= \frac{1}{1-x} .
  \end{align*}
\end{itemize}

Per una legge geometrica \(X\sim Ge(p)\) il parametro \(p\)
rappresenta \(P(X = 1)\).

La funzione di sopravvivenza è, per \(x\in\N^+\)
\[
  P(X>x) = P(C_1\cap \cdots \cap C_x) = P(C_1)\times\cdots\times
  P(C_x) = (1-p)^x
\]
per cui la funzione di ripartizione è
\[
  F_X(x) = P(X\leq x) = 1 - P(X > x) = 1 - (1-p)^x .
\]

Per \(x\in\R\) invece si ha
\[
  F_X(x) = \begin{cases}
    0 &\mbox{se } x < 1 \\
    1-(1-p)^{\floor{x}} &\mbox{se } x \geq 1 .
  \end{cases}
\]

Se \(0 < p < 1\) la v.c.\ \(X\sim Ge(p)\) può essere vista come il
tempo d'attesa nel tempo discreto che gode della proprietà di assenza
della memoria, ovvero per ogni \(s,t \in \N^+\)
\[
  P(X>s+t \mid X>s) \quad\mbox{non dipende da } s.
\]

\paragraph{Valore atteso}
\[
  \E(X) = \sfrac{1}{p}.
\]

\paragraph{Generatrice dei momenti e varianza}
\begin{align*}
  M_X(t) = \E(e^{tX})
  &= \sum_{x\in S_X} e^{tx} p_X(x)  &\\
  &= \sum_{x=1}^{+\infty} e^{t(x-1+1)}p(1-p)^{x-1} &\\
  &= pe^t \sum_{x=1}^{+\infty} e^{t(x-1)}(1-p)^{x-1} &\\
  &= pe^t \sum_{i=0}^{+\infty} (e^t(1-p))^i &\mbox{posto } i = x-1 \\
  &= \frac{ pe^t }{ 1-(1-p)e^t }
\end{align*}

Saltando alcuni passaggi si trova che
\[
  M'_X(t) \computedat{t=0}
  = \frac{pe^t}{\left( 1-(1-p)e^t \right)^2} \computedat{t=0}
  = \frac{1}{p}.
\]

Con calcoli lasciati al lettore\footnote{dal professore, non da
  me. Prendetevela con lui.}
\[
  \Var(X) = \frac{1-p}{p^2}.
\]

\section{Leggi di tipo continuo}

Si dice che \(X\) è una \textbf{v.c.\ univariata con legge di tipo
  continuo} se per ogni \(B\in\B_1\) si può esprimere \(P_X(B)\) in
forma integrale come
\[
  P_X(B) = P(X\in B) = \int_B p_X(x)dx = \int_a^b p_X(x) dx
\]
se \(B = [a,b]\) con \(a < b\), dove la funzione \(p_X(x)\) detta
funzione di densità di probabilità soddisfa le condizioni:
\begin{enumerate}
\item \(p_X(x) \geq 0\) per ogni \(x\in\R\)
\item \(\int_\R p_X(x) = \int_{-\infty}^{+\infty} p_X(x) dx = 1\)
\end{enumerate}

Una \(P_x\) definita in questo modo soddisfa gli assiomi di
Kolmogorov.


\subsection{Leggi uniformi continue}
Si dice che \(X\) ha legge uniforme continua in \((a,b)\), dove
\(a<b\), e si scrive \(X\sim U(a,b)\) se la f.d.p.\ di \(X\) è
\[
  p_X(x) = \begin{cases}
    1\over{b-a} & \mbox{se } x\in (a,b) \\
    0 & \mbox{se } x\not\in (a,b) .
  \end{cases}
\]
da cui valori atteso e varianza si ricavano facilmente come:
\begingroup
% inizia un nuovo group per ``localizzare'' il \addtolength in modo
% che non ``sballi'' gli align* seguenti.  Aggiunge un po' di spazio
% extra tra le righe per rendere le equazioni più leggibili IMO.
\addtolength{\jot}{1em}
\begin{align*}
  \E(X) &= \int_{-\infty}^{+\infty} xp_X(x) dx = \int_a^b x
          \frac{1}{b-a} dx = \frac{1}{b-a}\left[ \frac{1}{2}x^2 \right]_a^b =
          \frac{a+b}{2} \\
  \E(X^2) &= \int_a^b x^2 \frac{1}{b-a}dx = \frac{1}{b-a}\left[
            \frac{1}{3}x^3 \right]_a^b = \frac{a^2 + ab + b^2}{3} \\
  \Var(X) &= \E(X^2) - {(\E(X))}^2 = \cdots = \frac{{(b-a)}^2}{12}.
\end{align*}
\endgroup

\subsection{Leggi esponenziali}
Si dice che \(X\) ha legge esponenziale con parametro \(\lambda > 0\),
e si scrive \(X\sim Esp(\lambda)\), se la f.d.p.\ di \(X\) è
\[
  p_X(x) = \begin{cases}
    \lambda e^{-\lambda x} & \mbox{se } x \geq 0 \\
    0 & \mbox{se } x < 0
  \end{cases}
\]

La verifica della non-negatività è banale, e anche la condizione di
normalizzazione si verifica facilmente:
\begin{align*}
  \int_{-\infty}^{+\infty} p_X(x) dx &= \int_0^{+\infty} p_X(x) dx =
                                       \int_0^{+\infty} \lambda  e^{-\lambda x} dx =
                                       {\left[ -e^{-\lambda x} \right]}_0^{+\infty} \\
                                     & =- \lim_{x\to +\infty} e^{-\lambda x} - (-e^0) = 1.
\end{align*}

Calcolo del valore atteso:
\[
  \E(X) = \int_{-\infty}^{+\infty} x p_X(x) dx = \int_0^{+\infty} x
  \lambda e^{-\lambda x} dx = \frac{1}{\lambda} \int_0^{+\infty}
  \lambda x e^{-\lambda x} dx = 1
\]
dove l'ultimo integrale vale 1 (si calcola per parti dopo aver
applicato la sostituzione \(\lambda x = t\)).

Le leggi esponenziali sono la modellazione di \textit{default} per
tempi d'attesa.  Il risultato sul valore atteso dà un significato al
parametro \(\lambda\). Per un tempo d'attesa esponenziale,
\(\sfrac{1}{\lambda}\) è il valore medio dell'attesa. Quindi:
\[
  \lambda = \frac{1}{\mbox{media dell'attesa}} = \frac{1}{\E(X)}
\]

\paragraph{Funzione di ripartizione}
\[
  F_X(x) = \begin{cases}
    0 &\mbox{se } x < 0 \\
    1-e^{-\lambda x} &\mbox{se } x\geq 0
  \end{cases}
\]

\paragraph{Funzione generatrice dei momenti, valore atteso e varianza}
\begin{align*}
  M_X(t) = \E(e^{tX})
  &= \int_0^{+\infty} e^{tx} \lambda e^{-\lambda x} \dx \\
  &= \lambda \int_0^{+\infty} e^{-(\lambda - t)x} \dx \\
  &= \frac{\lambda}{\lambda - t} \int_0^{+\infty} (\lambda -t)e^{-(\lambda -t)}x \dx \\
  &= \frac{\lambda}{\lambda -t}
\end{align*}
da cui
\begin{align*}
  M'_X(t)
  &= - \left( 1 - \frac{t}{\lambda} \right)^{-2}
    \left(-\frac{1}{\lambda}\right) =
    \frac{1}{\lambda} \left( 1 - \frac{t}{\lambda} \right)^{-2} \\
  M''_X(t)
  &= \frac{-2}{\lambda} \left( 1-\frac{t}{\lambda}
    \right)^{-3} \left( -\frac{1}{\lambda} \right) =
    \frac{2}{\lambda^2} \left( 1 - \frac{t}{\lambda} \right)^{-3}
\end{align*}
da cui infine si ottiene
\begin{align*}
  \Var(X) = \E(X^2) - (\E(X))^2 = \frac 1 \lambda^2 .
\end{align*}

\subsection{Funzione tasso di guasto}
Sia \(T\) un tempo d'attesa, quindi una v.c.\ univariata con \(P(T\geq
0) = 1\) e legge continua. Siano date
\begin{description}
\item[funzione di ripartizione] \(F_T(t) = P(T \leq t)\)
\item[funzione di sopravvivenza] \(\FS_T(t) = 1-F_T(t) = P(T > t)\)
\item[f.d.p.] \(p_T(t) = \frac d{dt} F_T(t)\) supposta continua
  ovunque, salvo che in un numero finito di punti.
\end{description}

\begin{definition}
  Si dice funzione tasso di guasto di \(T\) (o \textit{hazard rate},
  \textit{failure rate}) la funzione \(r_T(\cdot)\) definita per i
  valori di \(t\) per cui \(F_T(t) < 1\) da
  \[
    r_T(t) = \frac{p_T(t)}{\FS_T(t)} = - \frac d{dt} \log\FS_T(t) .
  \]
\end{definition}

Nei punti in cui \(r_T(t)\) è continua, è anche proporzionale alla
probabilità che l'attesa, ancora viva al tempo \(t\), termini entro il
tempo \(t+\epsilon\), ossia nel tempuscolo immediatamente successivo a
\(t\).

Dalla funzione tasso di guasto si può determinare la funzione di
ripartizione e la funzione di densità di probabilità di \(T\),
infatti:
\[
  \int_0^t r_T(u) \du = -\log \FS_T(t)
\]
da cui si ottiene che per \(t > 0\)
\[
  F_T(t) = 1 - \exp \left\{ - \int_0^t r_T(u) \du \right\}
\]
e
\[
  p_T(t) = r_T(t) \times \exp \left\{ -\int_0^t r_T(u) \du \right\} .
\]

\subsection{Leggi di Weibull}

Un tempo d'attesa \(T_0\) ha legge di Weibull monoparametrica con
parametro di forma \(c>0\) se per \(t>0\) il suo tasso di guasto è
\[
  r_{T_0}(t) = c \times t^{c-1}
\]

Interessante è notare come \(Esp(1)\) faccia parte delle leggi di
Weibull, ma \(Esp(\lambda), \lambda \neq 1\) no.  Per questo motivo,
si introduce un secondo parametro alle leggi di Weibull:
\[
  r_T(t) = \lambda c (\lambda t)^{c-1} .
\]

Le esponenziali diventano un caso particolare \(Esp(\lambda) \sim W(1,
\lambda)\).

Si noti come \(r_{T_0}(t)\) sia decrescente per \(0 < c < 1\),
costante per \(c = 1\) e crescente se \(c > 1\).

Si può calcolare la funzione di ripartizione tenendo a mente che
\(\int_0^t r_T(u) \du = (\lambda t)^c\):
\[
  F_T(t) = 1 - \exp \{ -(\lambda t)^c \}
\]
e la corrispondente f.d.p.
\[
  p_T(t) = \lambda c (\lambda t)^{c-1} \exp\{ -(\lambda t)^c \} .
\]

\subsection{Leggi gamma}

Si tratta di un secondo modo per modellare tempi d'attesa nel continuo
con tasso di guasto monotono.

\begin{definition}
  Un tempo d'attesa \(T_0\) ha legge gamma monoparametrica con
  parametro di forma \(\alpha > 0\) se per \(t>0\) la sua funzione di
  densità di probabilità è
  \[
    p_{T_0}(t) = \frac 1{\Gamma(\alpha)} t^{\alpha-1}e^{-t} .
  \]
\end{definition}

La funzione \(\Gamma(\alpha)\) è la funzione gamma di Eulero ed è
definita dall'integrale convergente per \(\alpha > 0\)
\[
  \Gamma(\alpha) = \int_0^{+\infty} t^{\alpha -1}e^{-t} \dt.
\]
che è l'estensione della nozione di fattoriale per i numeri reali
positivi.

Anche in questo caso introdurre un parametro \textit{scala} \(\lambda
> 0\):
\[
  p_T(t) = \frac{\lambda^\alpha}{\Gamma(\alpha)} t^{\alpha -1}
  e^{-\lambda t} .
\]
in modo che le leggi esponenziali diventi un caso particolare di leggi
gamma \(Esp(\lambda) \sim Ga(1, \lambda)\).

Si dimostra (non qui) che \(r_{T_0}(t)\) è decrescente se \(0 < \alpha
< 1\), costante se \(\alpha = 1\) e crescente se \(\alpha > 1\)

\paragraph{Funzione generatrice dei momenti, valore atteso e varianza}
Alcuni passaggi sono stati omessi per brevità
\begin{align*}
  M_X(t) = \E(e^{tX})
  &= \int_0^{+\infty} e^{tx} \frac{\lambda^\alpha}{\Gamma(\alpha)} x^{\alpha -1} e^{-\lambda x} \dx \\
  &= \lambda^\alpha \times
    \int_0^{+\infty} \frac{x^{\alpha-1}}{\Gamma(\alpha)} e^{-(\lambda -t) x} \dx \\
  &= \frac{\lambda^\alpha}{(\lambda -t)^\alpha} \times
    \int_0^{+\infty} \frac{(\lambda -t)^\alpha}{\Gamma(\alpha)} e^{-(\lambda-t)x} \dx \\
  &= \frac{\lambda^\alpha}{(\lambda -t)^\alpha}
\end{align*}

Si ottiene
\begin{align*}
  \E(X) &= M'_X(0) = \frac{\alpha}{\lambda} \\
  \E(X^2) &= M''_X(0) = \frac{\alpha (\alpha +1)}{\lambda^2}
\end{align*}
da cui
\[
  \Var(X) = E(X^2) - (E(X))^2 = \frac{\alpha}{\lambda^2}.
\]

\paragraph{Proprietà additiva}
Sia \(X = (X_1, \dots, X_d)\) una v.c.\ multivariata con componenti
\(X_i\) indipendenti e legge marginale \(X_i \sim Ga(\alpha_i,
\lambda)\) dove \(\alpha_i > 0\) per ogni \(i = 1, \dots, n\) e
\(\lambda > 0\) per cui
\[
  M_{X_i}(t) = \left( 1 - \frac{t}{\lambda} \right)^{-\alpha_i}
\]
allora \(S = \sum_{i=1}^d X_i\) ha f.g.m.\ propria
\begin{align*}
  M_S(t) &= \prod_{i=1}^d M_{X_i}(t) \\
         &= \prod_{i=1}^d \left( 1 \ frac{t}{\lambda} \right)^{-\alpha_i} \\
         &= \left( 1 - \frac{t}{\lambda} \right)^{- \sum_{i=1}^d \alpha_i}
\end{align*}
da cui si evince che
\[
  S \sim Ga\left( \smallsum_{i=1}^d \alpha_i, \: \lambda \right) .
\]

\subsection{Leggi normali}

\begin{definition}[Legge normale standard]
  Una v.c.\ univariata \(Z\) con supporto \(S_Z = \R\) e f.d.p
  \[
    p_Z(z) = \frac{1}{\sqrt{2 \pi}} e^{- \frac{1}{2} z^2}
  \]
  è detta con legge normale standard, in breve \(Z \sim N(0,
  1)\).
\end{definition}

\begin{definition}[Legge normale con parametri]
  Una v.c.\ univariata \(X = \mu + \sigma Z\) con \(Z \sim N(0, 1)\) è
  detta normale con parametro di posizione \(\mu\) e parametro di
  scala \(\sigma\), in breve \(X \sim N(\mu, \sigma^2)\) e a f.d.p.
  \[
    % highlight that \sigma is not under the square root.
    p_X(x) = \frac{1}{\sqrt{2 \pi}\: \sigma} \exp \left\{ -\frac{1}{2} \left( \frac{x - \mu}{\sigma} \right)^2 \right\}
  \]
\end{definition}

Un'applicazione delle leggi normali è lo studio degli errori di
misurazione.  Si supponga di effettuare misurazioni ripetute con lo
stesso strumento di una certa quantità \(\mu\).  Le misure \(x_i\),
affette da errore, possono essere modellate come realizzazione della
variabile casuale \(X \fulfillment x_i\).  Gli stessi errori di
misurazione (ignoti) possono essere modellati come variabile casuale
\(Z \fulfillment z_i\).  In questo ultimo caso però conviene usare una
scala \textit{standard}\footnote{che quindi va \textit{scalata} di
  caso in caso con l'ausilio di un fattore \(\sigma\)}, e perciò
\[
  x_i = \mu + \sigma z_i \qquad i = 1, \dots, n \qquad \sigma \in \R
\]
da cui per confronto si può dedurre che \(X\) non è altro che una
trasformata
\[
  X = \mu + \sigma Z.
\]

\paragraph{Chiusura sotto trasformazioni affini}
Se \(X \sim N(\mu, \sigma^2)\) e \(T = a + bX\), con \(b \neq 0\)
allora
\[
  T \sim N\left( a+b\mu, b^2\sigma^2 \right) .
\]

La dimostrazione segue dall'applicazione della definizione di \(X\),
dal notare che \(Z\) è simmetrica \(Z \sim -Z\) e dal confronto:
\begin{align*}
  T &= a + bX &&\mbox{definizione di } T \\
    &\sim a + b(\mu + \sigma Z) &&\mbox{definizione di } X \\
    &\sim a + b\mu + b\sigma Z &&\\
    &\sim a + b\mu + \abs{b} \sigma Z &&\mbox{per simmetria} \\
    &\sim N(a+b\mu, b^2\sigma^2) &&\mbox{per confronto.}
\end{align*}

\paragraph{Funzione generatrice dei momenti}
\begin{align*}
  M_X(t)
  &= \E(e^{tX}) &&\mbox{def. di f.g.m.} \\
  &= \E\left( e^{t(\mu + \sigma Z)} \right) &&\mbox{def. di } X\\
  &= \E( e^{t\mu} e^{t\sigma Z} ) \\
  &= e^{t\mu} \E(e^{t \sigma Z}) &&\mbox{linearità di } \E \\
  &= e^{t\mu} M_Z(t\sigma) &&\mbox{def. di f.g.m.}
\end{align*}
esplicitiamo \(M_Z\)
\begin{align*}
  M_Z(t) = E(e^{tZ})
  &= \int_{-\infty}^{+\infty} e^{tz} \frac{1}{\sqrt{2 \pi}} e^{\frac{1}{2}z^2} \dz \\
  &= \int_{-\infty}^{+\infty} \frac{1}{\sqrt{2 \pi}}
    e^{-\frac{1}{2}(z^2 -2tz + t^2 - t^2)} \dz \\
  &= e^{\frac{1}{2}t^2}
    \explain{ \int_{-\infty}^{+\infty} \frac{1}{\sqrt{2\pi}} e^{\frac{1}{2}(z-t)^2} \dz}
    {si tratta della f.d.p. di \(N(t, 1)\) e quindi per la normalizzazione vale 1} \\
  &= e^{\frac{1}{2}t^2}
\end{align*}
e quindi
\[
  M_X(t) = e^{t\mu} e^{\frac{1}{2}(t\sigma)^2} = e^{t\mu + \frac{1}{2}t^2\sigma^2}
\]

Essendo \(M_Z(t)\) finita per ogni \(t\in\R\), \(Z\) ha f.g.m.\ propria.

Ora è possibile calcolare valore atteso e varianza:
\begin{align*}
  E(X) &= M'_X(t)\computedat{t=0} = \mu \\
  \E(X^2) &= M''_X(t)\computedat{t=0} = \sigma^2 + \mu^2 \\
  \Var(X) &= \E(X^2) - \left( \E(X) \right)^2 = \sigma^2 + \mu^2 - (\mu)^2 = \sigma^2.
\end{align*}

\paragraph{Proprietà additiva}
Se \(X \sim N(\mu_X, \sigma^2_X)\) e \(Y \sim N(\mu_Y, \sigma^2_Y)\)
sono indipendenti allora
\[
  S = X + 5 = \N(\mu_X + \mu_Y, \sigma^2_X + \sigma^2_Y) .
\]

\paragraph{Funzione di ripartizione}
La funzione di ripartizione di \(Z \sim N(0, 1)\) verrà indicata con
\(\Phi(z)\) definita come:
\[
  \Phi(z) = P(Z \leq z) = \int_{-\infty}^z \frac{1}{\sqrt{2\pi}} e^{\frac{1}{2}u^2} \du.
\]

È possibile ricondurre la f.r.\ di una v.c.\ normale
\(X \sim N(\mu, \sigma^2)\) a quella della normale standard \(Z\):
\begin{align*}
  F_X(x) &= P(X \leq x) &&\mbox{def. di f.r.} \\
         &= P(\mu + \sigma Z \leq x) &&\mbox{def. di } X \\
         &= P\left( Z \leq \frac{x-\mu}{\sigma} \right) &&\\
         &= \Phi\left( \frac{x-\mu}{\sigma} \right) .
\end{align*}

\end{document}

%%% Local Variables:
%%% mode: latex
%%% TeX-master: t
%%% End:
